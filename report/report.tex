\documentclass{article}

%% Packages
% --------
%	 Necessary
\usepackage{geometry} 						% geometry - page dimensions
\usepackage[parfill]{parskip}				% parskip - to use blank line to sep paragraphs
\usepackage{titling}						% title formatting
\usepackage{enumitem}

\usepackage{amsmath}						% AMS - math, fonts, symbols, theorem
\usepackage{amsfonts}
\usepackage{amssymb}
\usepackage{amsthm}
\usepackage{mathtools}						% mathtools - \coloneqq

\usepackage{tikz}							% tikz - graphs

\usepackage{listings}

\usepackage{hyperref}

% 	Additional
\usepackage{graphicx}						% graphicx - importing graphics from file
\usepackage[T1]{fontenc}
\usepackage[utf8]{inputenc}
%\usepackage{tgbonum}

\usepackage{forest}							% forest - easy trees
\usepackage{tikzsymbols}					% tikzsymbols - just adds some symbols
% tikzlibrary summary: 
% tex.stackexchange.com/questions/42611/list-of-available-tikz-libraries-with-a-short-introduction/491626
\usetikzlibrary{arrows.meta}				% arrows.meta - customizable arrow tips
\usetikzlibrary{positioning}
\usetikzlibrary{shapes}
\usetikzlibrary{shadows}
\usetikzlibrary{automata}
\usetikzlibrary{calc}
\usetikzlibrary{fit}

\usepackage{syntax}

\pagenumbering{gobble}

\tikzset{>={Latex[width=1.5mm, length=2mm]},%
	every state/.style={thick, fill=gray!10},
	initial text=$ $,%
	}
\usepackage{xcolor}							% xcolor - adds additional colors

\usepackage{marginnote}						% marginnote - see name

\usepackage{multirow}						% multirow - sort of a tabular environment of text
\usepackage{bigdelim}						% bigdelim - used with multirow once to make brackets on tables
\usepackage{array}							% array - extends arrary and tabular environment
\usepackage{makecell}						% makecell - adjust cell cizes within tabular environment

\usepackage[bottom]{footmisc}
% Customizations

\pretitle{\begin{center}\Large\bfseries}		% titling format
\posttitle{\par\end{center}\vskip 0cm}
\preauthor{\begin{center}\large}
\postauthor{\end{center}}
\predate{\par\normalsize\centering}
\postdate{\par}

\geometry{margin=1in}

\setlength{\droptitle}{-5em}

\newcommand{\xoverbrace}[2][\vphantom{\dfrac{A}{A}}]{\overbrace{#1#2}}
\newcommand{\xunderbrace}[2][\vphantom{\dfrac{A}{A}}]{\underbrace{#1#2}}

\newtheoremstyle{customnumber} % name
	{}% space above
	{}% space below
	{\normalfont}% body font
	{}% indent 
	{\bfseries}% head font
	{:}% punctuation between head/body
	{ }% space after head: " " = normal whitespace
	{\thmname{#1}\thmnote{ #3}}% head format

\theoremstyle{customnumber}
\newtheorem*{exercise}{Exercise}
\newenvironment{absolutelynopagebreak}
	{\par\nobreak\vfil\penalty0\vfilneg\vtop\bgroup}
	{\par\xdef\tpd{\the\prevdepth}\egroup\prevdepth=\tpd}
	
\theoremstyle{definition}
\newtheorem{defi}{Definition}

\newtheorem{theorem}{Theorem}

\newtheorem{axiom}{Axiom}

\newtheorem{ex}{Example}[section]

\newtheorem{prop}{Proposition}[section]

\newtheoremstyle{named}{}{}{\itshape}{}{\bfseries}{.}{ }{\thmnote{#3}}
\theoremstyle{named}
\newtheorem*{namedtheorem}{}

\newcommand{\stcomp}[1]{{#1}^\complement}

\newcommand{\norm}[1]{\left\lVert#1\right\rVert}

\newcommand{\var}{\mathrm{Var}}
\newcommand{\cov}{\mathrm{Cov}}

\lstset{tabsize=3, numbers=left, basicstyle=\ttfamily, escapeinside=~~, xleftmargin=1cm}
\let\origthelstnumber\thelstnumber
\makeatletter
\newcommand*\Suppressnumber{%
	\lst@AddToHook{OnNewLine}{%
		\let\thelstnumber\relax%
		\advance\c@lstnumber-\@ne\relax%
	}%
}
\newcommand*\Reactivatenumber{%
	\lst@AddToHook{OnNewLine}{%
		\let\thelstnumber\origthelstnumber%
		\advance\c@lstnumber\@ne\relax%
	}%
}
\makeatother

\def\smallqed{\hfill\smash{\scalebox{.5}{$\square$}}}


\DeclareMathOperator{\adj}{adj}

% if you need to pass options to natbib, use, e.g.:
%     \PassOptionsToPackage{numbers, compress}{natbib}
% before loading neurips_2019

% ready for submission
% \usepackage{neurips_2019}

% to compile a preprint version, e.g., for submission to arXiv, add add the
% [preprint] option:
     \usepackage[preprint]{neurips_2019}

% to compile a camera-ready version, add the [final] option, e.g.:
%    \usepackage[final]{neurips_2019}

% to avoid loading the natbib package, add option nonatbib:
%     \usepackage[nonatbib]{neurips_2019}

\usepackage[utf8]{inputenc} % allow utf-8 input
\usepackage[T1]{fontenc}    % use 8-bit T1 fonts
\usepackage{hyperref}       % hyperlinks
\usepackage{url}            % simple URL typesetting
\usepackage{booktabs}       % professional-quality tables
\usepackage{amsfonts}       % blackboard math symbols
\usepackage{nicefrac}       % compact symbols for 1/2, etc.
\usepackage{microtype}      % microtypography

\bibliographystyle{unsrtnat}

\title{GAN Dissecting aGAN}

% The \author macro works with any number of authors. There are two commands
% used to separate the names and addresses of multiple authors: \And and \AND.
%
% Using \And between authors leaves it to LaTeX to determine where to break the
% lines. Using \AND forces a line break at that point. So, if LaTeX puts 3 of 4
% authors names on the first line, and the last on the second line, try using
% \AND instead of \And before the third author name.

\author{%
  Tripp Isbell\\
  Auburn University\\
  \texttt{cai0004@auburn.edu} \\
  %% Add other stamps here
  \And
 Lalithya Kuntamukkala\\
 Auburn University\\
 \texttt{lzk0034@auburn.edu}\\
  \And
  James Lee\\
  Auburn University\\
  \texttt{jyl0003@auburn.edu}\\
  % examples of more authors
  % \And
  % Coauthor \\
  % Affiliation \\
  % Address \\
  % \texttt{email} \\
  % \AND
  % Coauthor \\
  % Affiliation \\
  % Address \\
  % \texttt{email} \\
  % \And
  % Coauthor \\
  % Affiliation \\
  % Address \\
  % \texttt{email} \\
  % \And
  % Coauthor \\
  % Affiliation \\
  % Address \\
  % \texttt{email} \\
}

\begin{document}

\maketitle

\begin{abstract}
We explore the GAN Dissection method and stuff.
\end{abstract}

NOTE: this is a big work in progress (particularly introduction and stuff), skip to sections 2,3,4 and particularly 4.2 to see current progress.

\section{Introduction}

%% placeholder text below

Much research in the field of Deep Learning has worked towards the goal of visualizing and understanding deep neural network architectures (cite much research). Introduce Network Dissection as a means of understanding semantic representation of Convolutional Neural Networks (cite net dissect). 

The field of generative models is a-boomin', and from it many unique models have emerged, including (cite unique models). A dominant/popular model for image generation is the Generative Adversarial Network (cite GANs) and many GAN variants (cite GAN variants, or maybe don't, probably don't blow all our citations so early). All of this research/progress has spawned new research into understanding generative models. And maybe go into some research on generative models (with citations). Since the generative models make use of neural networks, a lot of the research into network visualization and understanding can be applied. \citet{gandissect2019} adapt their method from \cite{netdissect2017} and build on it to understand and reason about GANs. 

cite spam to be copied:
\citet{netdissect2017}
\citet{gandissect2019}
\citet{progan2017}
\citet{stylegan2018}
\citet{gan2014}
\citet{synthesizing2016}
\citet{deepvis2015}

\section{GAN Dissection}

The paper presents a framework to visualize and understand GANs at the unit-, object-, and scene-level. It aims to understand how different objects and classes of objects are encoded by the GAN.

 A group of interpretable units that are closely related to object concepts are identified using a segmentation-based network dissection method. The causal effect of interpretable units is quantified by measuring the ability of interventions to control objects in the output .The contextual relationship between these units and their surroundings is examined by inserting the discovered object concepts into new images.

The structure of the featuremap is studied in two different phases
\begin{enumerate} 
\item Dissection: Identify the classes that have an explicit representation
\item Intervention: For the represented classes identified through dissection, we identify causal sets of units and measure causal effects between units and object classes by forcing sets of units on and off 
\end{enumerate}

\subsection{Applications}

In \citet{gandissect2019} during the continuous intervention phase used to identify sets of causal units, they add a regularization term to find a minimal set of causal units. We hypothesize that this minimization is equivalent to the 

\section{StyleGAN}

As an extended version of ProGan, StyleGAN allows for more control in the image synthetic process. This is possible in the ways that the StyleGAN generator was redesigned. Through the use of a mapping network of eight fully connected layers and getting a learned constant, this allows the model to generate a vector that can reduce the correlation between features. Then, learned affine transformations then specialize the vector(denoted at w in the paper) to styles that then control the adaptive instance normalization(AdaIn) where each feature map is normalized separately, and then scaled and biased using scalar components from the style. Lastly, "explicit noise inputs" are then introduced to generate stochastic detail. Compelling enough, this design led to improved generated image quality. Combined with that the generator architecture makes it possible to control the image synthesis through scale-specific modifications to the styles, we believe that this Gan model is a good candidate for our project because it is an interesting alternative extension to ProGan and that we are able to utilize tools that the Gan Dissection paper used as well. Lastly, we believe that studying and performing gan dissection on this model could lead to interesting results.

\section{Methods}

Our methods for this project involve researching other GAN models we think might yield interested results when studied under the GAN Dissection \cite{gandissect2019} framework. We find a pre-trained GAN and a segmented dataset as a candidate to be examined. We then apply the GAN dissection tool released by \citet{gandissect2019} at \url{http://github.com/CSAILVision/GANDissect} which carries out dissection on the specified model, and this basically carries out the heavy lifting. The tool also produces metrics and visualizations of sets of units with high agreement with specified concepts. 

\subsection{Progressive GANs}

We first replicate one of the experiments of \citet{gandissect2019}, using a Progressive GAN (\cite{progan2017}) trained on LSUN living room images, we use the GAN dissection tool to identify units which are highly activated by certain concepts. We analyze the results of this and make sure that they align with the results provided in \citet{gandissect2019}. The metrics provided by the tool will act as a sort of baseline to compare the results on other networks. \citet{gandissect2019} note that GAN dissection should not be used for qualitative comparisons across GAN architectures, but through comparison we seek to reason about why their differences might be so.

\subsection{Challenges / current progress}

As of right now (3/18), we've been working on getting the tool set up to run the dissection. We tried running with CUDA on a personal desktop GPU (running the above experiment) and after getting dependencies working, ran into memory issues running the tool. We plan to try again using a smaller batch size, but will likely need to use cloud computing or the Auburn GPU server. One of our group members has access to a GPU in their lab but the lab is closed down due to... circumstances. 

\subsection{StyleGAN }

Same stuff

\section{Results}

More to come.

\section{Analysis}

More to come.

\section{Discussion}

More to come.

\subsubsection*{Acknowledgments}

\bibliography{references}

\end{document}